% Modified for dvips

\documentclass[12pt,dvips]{article}
\usepackage{color}

%Set up hyperref for pdflatex

% \usepackage[bookmarks,
%             bookmarksopen,
%             pdftex=true]{hyperref}
\definecolor{Orange}{cmyk}{0,0.61,0.87,0}
\newcommand{\mysection}[1]{\color{blue}
            \section{#1} \normalcolor}
\newcommand{\mysubsection}[1] {\color{green}
            \subsection{#1} \normalcolor}
\newcommand{\mysubsubsection}[1]{\color{Orange}
            \subsubsection{#1} \normalcolor}
\newcommand{\myitem}[1]{\item{\bf \color{Violet} #1 \normalcolor}}


\addtolength{\oddsidemargin}{-0.5in} 
\addtolength{\textheight}{1.0in}
\addtolength{\textwidth}{0.5in} 
\setlength{\parindent}{0in}
%\renewcommand{\baselinestretch}{1.5}
\newcommand{\meta}[1]{{\ensuremath{<}}#1{\ensuremath{{>}}}}
\newcommand{\us}{\underline{ }}
\newcommand{\inrange}[2]{\hfill Input Range: \ensuremath{\left[ #1:#2
\right]}\\}
\newcommand{\inrangeopen}[1]{\hfill Input range: \ensuremath{\left( #1 \right)}\\}
\newcommand{\searchrange}[1]{\hfill Search range: 
    \ensuremath{\left[ #1 \right]}\\}
\newcommand{\searchrangeopen}[1]{\hfill Search range: 
    \ensuremath{\left( #1 \right)}\\}
\newcommand{\rtiny}{10^{-37}}
\renewcommand{\labelitemi}{\ensuremath{\bullet}}
\renewcommand{\labelitemii}{\ensuremath{\circ}}
\renewcommand{\labelitemiii}{\ensuremath{\odot}}
\definecolor {blue} {rgb} {0, 0, 1}
\definecolor {red} {rgb}  {1, 0, 0}
\definecolor {green} {rgb} {0, 1, 0}
\definecolor {white} {gray} {1}
\definecolor {black} {gray} {0}
\definecolor {cyan} {cmyk} {1, 0, 0 , 0}
\definecolor {magenta} {cmyk} {0, 1, 0, 0}
\definecolor {yellow} {cmyk} {0, 0, 1, 0}
\definecolor{Orange}        {cmyk}{0,0.61,0.87,0}
\definecolor{Violet}        {cmyk}{0.79,0.88,0,0}

\begin{document}

% Title Page

\vspace*{\fill}

{\Large \centering \textcolor {red} {RANDLIB90\\
\textcolor {blue}{Fortran 95 Routines for Random Number Generation}\\
\textcolor {green} {User's Guide}\\}}

\vspace{0.5in}

{\large \centering \textcolor{Orange}{Barry W. Brown\\
James Lovato\\}}

\vspace*{\fill}

\mysection{Technicalities}

\mysubsection{Obtaining the Code}

The source for this code (and all code written by this group) can be
obtained from the following URL:\\

{\bf http://odin.mdacc.tmc.edu/anonftp/\\}

\mysubsection{Legalities}

We place  our efforts  in writing this  package in the  public domain.
However,  code from  ACM publications  is  subject to  the ACM  policy
(below).

\mysubsection{References}

\mysubsubsection{Base Generator}

The base generator and all code in the ecuyer\_cote\_mod come from the
reference:

P.   L'Ecuyer  and S.  Cote.  (1991)  ``Implementing  a Random  Number
Package   with  Splitting   Facilities.''   {\em ACM   Trans.  on
Math. Softw. 17}:1, pp 98-111.

We transliterated the Pascal of the reference to Fortran 95.

\mysubsection{The Beta Random Number Generator}

R. C.   H. Cheng (1978)  ``Generating  Beta Variates  with Nonintegral
Shape  Parameters.''   {\em  Communications of   the  ACM, 21}:317-322
(Algorithms B and BC)

\mysubsection{The Binomial Random Number Generator}

Kachitvichyanukul, V. and   Schmeiser,
B.  W.  (1988)    ``Binomial   Random   Variate Generation.''     {\em
Communications of the ACM, 31}: 216.  (Algorithm BTPE.)

\mysubsection{The Standard Exponential Random Number Generator}

Ahrens,  J.H.  and Dieter, U.  (1972)  ``Computer Methods for Sampling
from the Exponential and  Normal Distributions.''  {\em Communications
of the ACM, 15}: 873-882.  (Algorithm SA.)

\mysubsection{The Standard Gamma Random Number Generator}

Ahrens, J.H. and  Dieter, U. (1982) ``Generating  Gamma Variates by  a
Modified Rejection Technique.''  {\em Communications  of the ACM, 25}:
47-54.  (Algorithm SA.)

\mysubsection{The Standard Normal Random Number Generator}

Ahrens, J.H.  and Dieter, U.  (1973) ``Extensions of Forsythe's Method
for    Random Sampling   from   the     Normal  Distribution.''   {\em
Math. Comput., 27}:927 - 937.  Algorithm FL (method=5)

\mysubsection{ACM Policy on Use of Code}

Here  is the  software Policy of  the  ACM.

\begin{quote}

     Submittal of  an  algorithm    for publication  in   one of   the  ACM
     Transactions implies that unrestricted use  of the algorithm within  a
     computer is permissible.   General permission  to copy and  distribute
     the algorithm without fee is granted provided that the copies  are not
     made  or   distributed for  direct   commercial  advantage.    The ACM
     copyright notice and the title of the publication and its date appear,
     and  notice is given that copying  is by permission of the Association
     for Computing Machinery.  To copy otherwise, or to republish, requires
     a fee and/or specific permission.

     Krogh, F.  (1997) ``Algorithms  Policy.''  {\em ACM  Tran.  Math.
     Softw.  13}, 183-186.

\end{quote}

Here is our standard disclaimer.

\begin{quote}

{\centering NO WARRANTY\\}

WE PROVIDE ABSOLUTELY  NO WARRANTY  OF ANY  KIND  EITHER  EXPRESSED OR
IMPLIED,  INCLUDING BUT   NOT LIMITED TO,  THE  IMPLIED  WARRANTIES OF
MERCHANTABILITY AND FITNESS FOR A PARTICULAR PURPOSE.  THE ENTIRE RISK
AS TO THE QUALITY AND PERFORMANCE OF THE PROGRAM IS  WITH YOU.  SHOULD
THIS PROGRAM PROVE  DEFECTIVE, YOU ASSUME  THE COST  OF  ALL NECESSARY
SERVICING, REPAIR OR CORRECTION.

IN NO  EVENT  SHALL THE UNIVERSITY  OF TEXAS OR  ANY  OF ITS COMPONENT
INSTITUTIONS INCLUDING M. D.   ANDERSON HOSPITAL BE LIABLE  TO YOU FOR
DAMAGES, INCLUDING ANY  LOST PROFITS, LOST MONIES,   OR OTHER SPECIAL,
INCIDENTAL   OR  CONSEQUENTIAL DAMAGES   ARISING   OUT  OF  THE USE OR
INABILITY TO USE (INCLUDING BUT NOT LIMITED TO LOSS OF DATA OR DATA OR
ITS ANALYSIS BEING  RENDERED INACCURATE OR  LOSSES SUSTAINED  BY THIRD
PARTIES) THE PROGRAM.

(Above NO WARRANTY modified from the GNU NO WARRANTY statement.)
\end{quote}


\pagebreak

\mysection{Introduction: The ecuyer\_cote\_mod module}

The base random generator for this set of programs contains 32 virtual
random number generators.  Each generator can provide 1,048,576 blocks
of numbers, and each block  is of length 1,073,741,824.  Any generator
can be  set to the  beginning or  end of the  current block or  to its
starting  value.  The  methods are  from the  paper  cited immediately
below, and  most of the code  is a transliteration from  the Pascal of
the paper into Fortran.


Most users won't need  the sophisticated capabilities of this package,
and will desire a single generator.  This single generator (which will
have  a non-repeating  length of  $2.3 \times  10^18$ numbers)  is the
default.

\mysection{Initializing the Random Number Generator}

You almost  certainly want  to start the  random number  generation by
setting the seed.  (Defaults are provided  if the seed is not set, but
the defaults will  produce the same set of numbers on  each run of the
program which is generally not what is wanted.)

Make   sure  that  the  program unit  that   sets  the seed has\\  
USE user\_set\_generator at its top.

The most straightforward  way  of  setting  the  values  need  by  the
generator is a call to USER\_SET\_ALL.
\vspace{0.2in}

SUBROUTINE USER\_SET\_ALL( SEED1, SEED2, GENERATOR)

\vspace{0.2in}
{\centering  ARGUMENTS\\}
\vspace{0.1in}

\begin{itemize}

\myitem{INTEGER, INTENT(IN) :: SEED1} The first number used to set the
generator.

\myitem{INTEGER, INTENT(IN) :: SEED2} The second number used to set the
generator.

\myitem{INTEGER, INTENT(IN), OPTIONAL :: GENERATOR} The current
generator.  If this argument is not present, the current generator is
set to 1.
\inrange{1}{32}

\end{itemize}

It is easier to remember an English phrase, such as your name, than it
is  to  remember two large  integer   values.   The following  routine
prompts interactively for a phrase (non-blank character string -- only
the first 80  characters are used)  and hashes the  phrase  to set the
seeds of the random number generator.  The current generator is set to
1.

\vspace{0.2in}
SUBROUTINE INTER\_PHRASE\_SET\_SEEDS()
\vspace{0.2in}

This routine calls the following routine which is the non-interactive
version of the same action.

\vspace{0.2in}
SUBROUTINE PHRASE\_SET\_SEEDS(PHRASE)\\
\hspace{0.5in} CHARACTER(LEN=*), INTENT(IN) :: PHRASE
\vspace{0.2in}

This in turn calls the lowest level routine:

\vspace{0.2in}
CALL PHRASE\_TO\_SEED( PHRASE, SEED1, SEED2 )\\
\hspace{0.5in} CHARACTER(LEN=*), INTENT(IN) :: PHRASE\\
\hspace{0.5in} INTEGER, INTENT(OUT) :: SEED1, SEED2
\vspace{0.2in}

In  this call,  phrase is  an  (input) character  string of  arbitrary
length and seed1 and seed2 are two (output) integer values that may be
used to initialize the seeds of the random number generator.

Setting  the seeds  to  a fixed  value   is  extremely useful  in  the
debugging  phase of a project  where exact runs  are to be replicated.
However, there are occasions  in which a  random appearing sequence is
desired -- an  example is computer games.   The following routine sets
the seeds of the generator using the system clock.  

\vspace{0.2in}
SUBROUTINE TIME\_SET\_SEEDS()
\vspace{0.2in}

If the    computer doesn't have a  clock   (unlikely) then the routine
queries  the user for a  phrase to use to set  the seeds.  The current
generator is set to 1.

The  next  described routine  allows  the programmer to select setting
seeds by time or from a phrase entered interactively.

\vspace{0.2in}
SUBROUTINE SET\_SEEDS(WHICH)\\
\hspace{0.5in} INTEGER, OPTIONAL, INTENT(IN) :: WHICH
\vspace{0.2in}

If WHICH  is not present, the routine  queries the  user as to whether
time or a phrase  is to be  used to set the random  number seeds.   If
WHICH is present  in the calling list and  equals 1 then the seeds are
set from the time;  if WHICH is present and   not equal to 1  then the
seeds are set for a phrase promted from the user..

The current generator is set through 

\vspace{0.2in}
SUBROUTINE SET\_CURRENT\_GENERATOR(G)\\
\hspace{0.5in} INTEGER, INTENT(IN) :: G
\vspace{0.2in}

The integer argument G sets the number of the current generator to its
value which must be between 1 and 32.

\pagebreak

\mysection{random\_beta\_mod}

      FUNCTION RANDOM\_BETA(A,B)

\mysubsection{The Distribution}

The density of the beta distribution is defined on $x$ in $[0,1]$ and
is proportional to
\[ x^a (1-x)^b \]

\mysubsection{Arguments}

\begin{itemize}

\myitem{REAL, INTENT(IN) :: A.} The first parameter of the beta
  distribution (a above).
\inrange{\rtiny}{\infty}

\myitem{REAL, INTENT(IN) :: B.} The second parameter of the beta
  distribution (b above).
\inrange{\rtiny}{\infty}

\end{itemize}

\pagebreak

\mysection{random\_binomial\_mod}

      REAL FUNCTION RANDOM\_BINOMIAL(N,PR)

\mysubsection{The Distribution}

The density of the binomial distribution provides the probability of S
successes in N independent trials, each with probability of success
PR.

The density is proportional to
\[ PR^S (1-PR)^{N-S} \]

The routine returns values of S drawn from this distribution.

\mysubsection{Arguments}

\begin{itemize}

\myitem{INTEGER, INTENT(IN) :: N.}  The number of binomial trials.
\inrange{0}{\infty}


\myitem{REAL, INTENT(IN) :: PR.}  The probability of success at each
  trial.
\inrange{0}{1}

\end{itemize}

\pagebreak

\mysection{random\_chisq\_mod}

      REAL FUNCTION RANDOM\_CHISQ(DF)

\mysubsection{The Distribution}

The chi-squared distribution is the  distribution of the sum of squares
of DF independent unit (mean 0, sd 1) normal deviates.

The density  is defined on $x$ in $[0,\infty)$ and
is proportional to
\[ x^{(DF-2)/2} \exp(-x/2 \]

\mysubsection{Argument}

\begin{itemize}

\myitem{REAL, INTENT(IN) :: DF.} The degrees of freedom of the
  chi-square distribution.
\inrange{\rtiny}{\infty}

\end{itemize}

\pagebreak

\mysection{random\_exponential\_mod}

      REAL FUNCTION RANDOM\_EXPONENTIAL(AV)

\mysubsection{The Distribution}

The density is:

\[ 1/AV \exp(-x/AV) \]

\mysubsection{Argument}

\begin{itemize}

\myitem{REAL, INTENT(IN) :: AV.} The mean of the exponential; 
see density above.
\inrange{\rtiny}{\infty}

\end{itemize}

\pagebreak

\mysection{random\_gamma\_mod}

REAL FUNCTION RANDOM\_GAMMA(scale,shape)

\mysubsection{The Distribution}

The density of the GAMMA distribution is proportional to:

\[ (x/SCALE)^{SHAPE-1} \exp(-x/SCALE) \]

\mysubsection{Arguments}

\begin{itemize}

\myitem{REAL (dpkind), INTENT(IN) :: SHAPE.}  The shape parameter of the
distribution (See above.)\\
\inrange{\rtiny}{\infty}

\myitem{REAL (dpkind), INTENT(IN) :: SCALE.}  The scale parameter of the
distribution.\\ \inrange{\rtiny}{\infty}

\end{itemize}

\pagebreak

\mysection{random\_multinomial\_mod}

      SUBROUTINE RANDOM\_MULTINOMIAL(N,P,NCAT,IX)

\mysubsection{The Distribution}

Each observation falls into one category where the categories are
numbered $1 \ldots NCAT$.  The observations are independent and the
probability that the outcome will be category $i$ is $P(i)$.  There
are N observations altogether, and the number falling in category $i$
is IX(i).

\mysubsection{Arguments}

\begin{itemize}

\myitem{INTEGER, INTENT(IN) :: N} The number of observations from the 
multinomial distribution.
\inrange{1}{\infty}

\myitem{REAL,   DIMENSION(NCAT-1),  INTENT(IN)   ::  P}   P(i)   is  the
probability that an observation falls into category i.  NOTE: The P(i)
must be  non-negative and  less than  or equal to  1.  Only  the first
NCAT-1 P(i) are  used, the final one is  obtained from the restriction
that all NCAT P's must add to 1.

\myitem{INTEGER, INTENT(IN) :: NCAT}  The number of outcome categories.
NCAT must be at least 2.

\myitem{INTEGER, INTENT(OUT), IX(NCAT)} 

\end{itemize}

\pagebreak

\mysection{random\_multivariate\_normal\_mod}

\mysection{The Distribution}

The multivariate normal density is:

\[ (2 \pi)^{-(k/2)} |COVM|^{-1/2} \exp(\left[ -(1/2)(X-MEANV)^T
COVM^{-1} (X-MEANV)\right] ) \]
where superscript T is the transpose operator.

\mysection{Use of Routines}

First, CALL SET\_RANDOM\_MULTIVARIATE\_NORMAL.  This routine processes
and saves information needed to generate multivariate normal deviates.
Successive calls to RANDOM\_MULTIVARIATE\_NORMAL generate deviates
from values specified in the most recent SET routine.

SUBROUTINE SET\_RANDOM\_MULTIVARIATE\_NORMAL(MEANV,COVM,P)

\mysubsection{Arguments}

\begin{itemize}

\myitem{REAL, INTENT(IN), DIMENSION(:) :: MEANV.} The mean of the
multivariate normal.  The size of MEANV must be at least P; if it is
more than P, entries following the P'th are ignored.

\myitem{REAL,  INTENT(IN),   DIMENSION(:,:)  ::  COVM.}    The  variance
covariance matrix of the multivariate normal.  Both dimensions must be
at  least P;  if the  matrix  is bigger  than PXP,  extra entries  are
ignored.

\myitem{INTEGER,  INTENT(IN) ::  P.} The  dimension of  the multivariate
normal deviates to be generated.

\end{itemize}

SUBROUTINE RANDOM\_MULTIVARIATE\_NORMAL(X)

\mysubsection{Arguments}

\begin{itemize}

\myitem{REAL, INTENT(OUT), DIMENSION(:) :: X.}  Contains the generated
multivariate normal deviate.  NOTE.  No check on the size of X is made
-- the programmer must assure that it is big enough.

\end{itemize}

\pagebreak

\mysection{random\_nc\_chisq\_mod}

FUNCTION random\_nc\_chisq(df,pnonc)

\mysubsection{The Distribution}

The noncentral  chi-squared distribution is the sum  of DF independent
normal  distributions  with  unit  standard deviations,  but  possibly
non-zero  means .  Let  the mean  of the  $i$th normal  be $\delta_i$.
Then PNONC = $ \sum_i \delta_i$.

\mysubsection{Arguments}

\begin{itemize}

\myitem{REAL, INTENT(IN) :: DF.} The degrees of freedom of the
noncentral chi-squared.

\myitem{REAL, INTENT(IN) :: PNONC.} The noncentrality parameter of the
noncentral chi-squared.

\end{itemize}

\pagebreak

\mysection{random\_negative\_binomial\_mod}

FUNCTION random\_negative\_binomial(s,pr)

\mysubsection{The Distribution}

The  density  of  the  negative  binomial  distribution  provides  the
probability  of  precisely  F  failures  before the  S'th  success  in
independent binomial trials, each with probability of success PR.

The density is 

\[ \left( \begin{array}{c} F+S-1\\ S-1 \end{array} \right) PR^S (1-PR)^F \]

This routine returns a random value of F given values of S and P.

\mysubsection{Arguments}

\begin{itemize}

\myitem{INTEGER, INTENT(IN) :: S.}  The number of succcesses after which
the number of failures is not counted.

\myitem{REAL, INTENT(IN) :: PR.} The probability of success in each
binomial trial.

\end{itemize}

\pagebreak

\mysection{random\_normal\_mod}

FUNCTION random\_normal(mean,sd)

\mysubsection{The Distribution}

The density of the normal distribution is proportional to

\[ \exp\left(-\frac{(X-MEAN)^2}{2 SD^2}\right) \]

\mysubsection{Arguments}

\begin{itemize}

\myitem{REAL, INTENT(IN) :: MEAN.} The mean of the normal distribution.

\myitem{REAL, INTENT(IN) :: SD.} The standard deviation of the normal
distribution.

\end{itemize}

\pagebreak

\mysection{random\_permutation\_mod}

SUBROUTINE RANDOM\_PERMUTATION(ARRAY,LARRAY)

\mysubsection{Function}

Returns a random permutation of (integer) array.

\mysubsection{Arguments}

\begin{itemize}

\myitem{INTEGER, INTENT(INOUT) :: ARRAY.}  On input an array of integers
(usually 1..larray).  On output a random permutation of this array.

\myitem{INTEGER, INTENT(IN),  OPTIONAL:: LARRAY.} If  present the length
of ARRAY to  be used .  If absent,  all of ARRAY is used  -- i.e., the
SIZE function is invoked to determine the dimensioned size of ARRAY

\end{itemize}

\pagebreak

\mysection{random\_poisson\_mod}

FUNCTION random\_poisson(lambda)

\mysubsection{The Distribution}

The density of the Poisson distribution (probability of observing S
events) is:

\[ \frac{LAMBDA^S}{S!} \exp(-LAMBDA) \]

\mysubsection{Argument}

\begin{itemize}

\myitem{REAL, INTENT(IN) :: LAMBDA.} The mean of the Poisson
distribution.

\end{itemize}

\pagebreak

\mysubsection{random\_uniform\_integer}

FUNCTION RANDOM\_UNIFORM\_INTEGER(LOW,HIGH)

\mysubsection{Function}

Returns a random integer between LOW and HIGH (inclusive of the limits).

\mysubsection{Arguments}

\begin{itemize}

\myitem{INTEGER, INTENT(IN) :: LOW.}  Low limit of uniform integer.

\myitem{INTEGER, INTENT(IN) :: HIGH.}  High limit of uniform integer.

\end{itemize}

\pagebreak

\mysection{random\_uniform\_mod}

FUNCTION random\_uniform(low,high)

\mysubsection{Function}

Returns a random real between LOW and HIGH.

\mysubsection{Arguments}

\begin{itemize}

\myitem{REAL, INTENT(IN) :: LOW.}  Low limit of uniform real.

\myitem{REAL, INTENT(IN) :: HIGH.}  High limit of uniform real.

\end{itemize}

\pagebreak

\mysection{The ``Standard'' Generator Modules}

It is not really intended that these modules be used directly.
However, for completeness, here is a listing of them.

\mysubsection{random\_standard\_exponential\_mod}

FUNCTION random\_standard\_exponential()

\mysubsubsection{Function}

Returns a random value from an exponential distribution with mean one.

\mysubsection{random\_standard\_normal\_mod}

FUNCTION random\_standard\_normal()

\mysubsubsection{Function}

Returns a random value from a normal distribution with mean 0 and
standard deviation one.

\mysubsection{random\_standard\_uniform\_mod}

FUNCTION random\_standard\_uniform()

\mysubsubsection{Function}

Returns a random value from a uniform  distribution on zero to one.

\mysection{Advanced Use of the ecuyer\_cote\_mod Module} 

Recall the following information from the Introduction.

\begin{quote}
The base random generator for this set of programs contains 32 virtual
random number generators.  Each generator can provide 1,048,576 blocks
of numbers, and each block  is of length 1,073,741,824.  Any generator
can be  set to the  beginning or  end of the  current block or  to its
starting  value.
\end{quote}

\mysubsection{Setting the Virtual Random Number Generator}

SUBROUTINE set\_current\_generator(g)

Sets  the current virtual generator to  the integer  value g, where $1
\le g \le 32$.   Before this routine  is called, the current generator
is 1.

\mysubsubsection{Arguments}

\begin{itemize}

\myitem{INTEGER, INTENT(IN) :: G.}  The number of the virtual  generator
to be used.

\end{itemize}

\mysubsection{Getting the Virtual Random Number Generator}

      SUBROUTINE get\_current\_generator(g)

\mysubsubsection{Arguments}

\begin{itemize}

\myitem{INTEGER, INTENT(OUT) :: G.}  The number of the virtual  generator
currently in use.

\end{itemize}

\mysubsection{Reinitializing the Current Virtual Generator}

      SUBROUTINE reinitialize\_current\_generator(isdtyp)

Reinitializes the state of the current generator.

\mysubsubsection{Arguments}

\begin{itemize}

\myitem{INTEGER, INTENT(IN) :: ISDTYP.}  The initialization action to be
performed.

\begin{description}

\myitem{-1} Sets the seeds of the current generator to their value at
the beginning of the current run.

\myitem{0} Sets the seeds to the first value of the current block of seeds.

\myitem{1} Sets the seeds to the first value of the next block of seeds.

\end{description}

\end{itemize}

\mysection{Getting the Current Seeds of the Virtual Random Number Generator}

      SUBROUTINE get\_current\_seeds(iseed1,iseed2)

\mysubsection{Arguments}

Gets the current value of the two integer seeds of the current generator.

\begin{itemize}

\myitem{INTEGER, INTENT(OUT) :: ISEED1.}  The value of the first seed.

\myitem{INTEGER, INTENT(OUT) :: ISEED2.}  The value of the second seed.

\end{itemize}

\mysection{Setting the Current Seeds of the Virtual Random Number Generator}

      SUBROUTINE set\_current\_seeds(iseed1,iseed2)

\mysubsection{Arguments}

Sets the current value of the two integer seeds of the current generator.



\begin{itemize}

\myitem{INTEGER, INTENT(IN) :: ISEED1.}  The value of the first seed.

\myitem{INTEGER, INTENT(IN) :: ISEED2.}  The value of the second seed.

\end{itemize}

\mysection{Producing  Antithetic Random    Numbers from  the   Current
Generator}

      SUBROUTINE set\_antithetic(qvalue)

Sets the current generator to produce antithetic values  or not.  If X
is  the value normally returned  from  a uniform  [0,1] random  number
generator then  1  - X is the   antithetic value. If  X  is  the value
normally returned from a uniform [0,N] random  number generator then N
- 1 -  X is the antithetic value.   All generators  are initialized to
NOT generate antithetic values.

\mysubsection{Arguments}

\begin{itemize}

\myitem{LOGICAL,  INTENT (IN) ::  qvalue} If .TRUE., antithetic values
are generated.   If  .FALSE.,   non-antithetic  (ordinary) values  are
generated.

\end{itemize}

\mysection{Bottom Level Support Routines}

      SUBROUTINE advance\_state(k)

Advances the current state of the seeds of the current virtual random
number generator by $2^k$ values.

\begin{itemize}

\myitem{INTEGER,  INTENT (IN) ::  k} The seeds of the current virtual
generator are set ahead by $s^k$ values.

\end{itemize}

      FUNCTION multiply\_modulo(a,s,m)

Returns (A*S) modulo M.

\begin{itemize}

\myitem{INTEGER, INTENT(IN) :: A}  First component of product.  Must
be $<$ M.

\myitem{INTEGER, INTENT(IN) :: S}  Second component of product.  Must
be $\le$ M.

\myitem{INTEGER, INTENT(IN) :: M}  Integer with respect to which the
modulo operation is performed.

\end{itemize}

\end{document}


